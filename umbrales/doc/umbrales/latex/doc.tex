\documentclass[11pt]{article} 
\usepackage[latin1]{inputenc} 
\usepackage[T1]{fontenc} 
\usepackage{textcomp}
\usepackage{fullpage} 
\usepackage{url} 
\usepackage{ocamldoc}
\begin{document}
\tableofcontents
\section{Module {\tt{Conexion\_db}} : modulo para realizar la conexion a la base de datos}
\label{module:Conexion-underscoredb}\index{Conexion-underscoredb@\verb`Conexion_db`}




\ocamldocvspace{0.5cm}



\label{exception:Conexion-underscoredb.E}\begin{ocamldoccode}
exception E of string
\end{ocamldoccode}
\index{E@\verb`E`}




\label{val:Conexion-underscoredb.get}\begin{ocamldoccode}
val get : {\textquotesingle}a option -> {\textquotesingle}a
\end{ocamldoccode}
\index{get@\verb`get`}




\label{val:Conexion-underscoredb.get-underscorerows}\begin{ocamldoccode}
val get_rows :
  string ->
  (Sqlite3.row_not_null -> Sqlite3.headers -> unit) -> Sqlite3.db -> unit
\end{ocamldoccode}
\index{get-underscorerows@\verb`get_rows`}
\begin{ocamldocdescription}
funcion para obtener las entradas de una tabla y aplicar una funcion a cada
fila que regresa como resultado


\end{ocamldocdescription}




\label{val:Conexion-underscoredb.open-underscoredb-underscoreconection}\begin{ocamldoccode}
val open_db_conection : string -> Sqlite3.db
\end{ocamldoccode}
\index{open-underscoredb-underscoreconection@\verb`open_db_conection`}
\begin{ocamldocdescription}
abre la conexion a la base de datos


\end{ocamldocdescription}




\label{val:Conexion-underscoredb.close-underscoredb-underscoreconection}\begin{ocamldoccode}
val close_db_conection : Sqlite3.db -> unit
\end{ocamldoccode}
\index{close-underscoredb-underscoreconection@\verb`close_db_conection`}
\begin{ocamldocdescription}
Cierra la conexion a la base de datos


\end{ocamldocdescription}


\section{Module {\tt{Grafica}} : Modulo para representar graficas de ciudades}
\label{module:Grafica}\index{Grafica@\verb`Grafica`}




\ocamldocvspace{0.5cm}



\label{exception:Grafica.ErrorGrafica}\begin{ocamldoccode}
exception ErrorGrafica of string
\end{ocamldoccode}
\index{ErrorGrafica@\verb`ErrorGrafica`}
\begin{ocamldocdescription}
Exepcion para indicar errores en las operaciones de la grafica


\end{ocamldocdescription}




\label{type:Grafica.ciudad}\begin{ocamldoccode}
type ciudad = {\char123}
  id : int ;
  nombre : string ;
  pais : string ;
  poblacion : int ;
  latitud : float ;
  longitud : float ;
  vecinos : (int, float) Hashtbl.t ;
{\char125}
\end{ocamldoccode}
\index{ciudad@\verb`ciudad`}
\begin{ocamldocdescription}
record para guradar la informacion de una ciudad


\end{ocamldocdescription}




\label{val:Grafica.tamano}\begin{ocamldoccode}
val tamano : int Pervasives.ref
\end{ocamldoccode}
\index{tamano@\verb`tamano`}




\label{val:Grafica.orden}\begin{ocamldoccode}
val orden : int Pervasives.ref
\end{ocamldoccode}
\index{orden@\verb`orden`}




\label{val:Grafica.initgraf}\begin{ocamldoccode}
val initgraf : int -> ciudad array
\end{ocamldoccode}
\index{initgraf@\verb`initgraf`}
\begin{ocamldocdescription}
funcion que inicializa la grafica. Las graficas se representan como un arreglo de referencias
  a listas. Cada entrada del arreglo representa un verice y cada lista contiene las adyacencias
  del nodo


\end{ocamldocdescription}




\label{val:Grafica.agrega}\begin{ocamldoccode}
val agrega : {\textquotesingle}a array -> {\textquotesingle}a -> unit
\end{ocamldoccode}
\index{agrega@\verb`agrega`}
\begin{ocamldocdescription}
funcion para agregar una ciudad a la grafica


\end{ocamldocdescription}




\label{val:Grafica.conectados}\begin{ocamldoccode}
val conectados : ciudad array -> int -> int -> bool
\end{ocamldoccode}
\index{conectados@\verb`conectados`}
\begin{ocamldocdescription}
Valida si es posible conectar dos nodos o si ya existe una arista que los une en esa direccion


\end{ocamldocdescription}




\label{val:Grafica.conecta}\begin{ocamldoccode}
val conecta : ciudad array -> int -> int -> float -> unit
\end{ocamldoccode}
\index{conecta@\verb`conecta`}
\begin{ocamldocdescription}
conecta dos nodos de una grafica. Al conecta los nodos i y j se agrega a la entrada i del arreglo
el par (j,p) donde p es el peso de la arista


\end{ocamldocdescription}




\label{val:Grafica.getPeso}\begin{ocamldoccode}
val getPeso : ciudad array -> int -> int -> float
\end{ocamldoccode}
\index{getPeso@\verb`getPeso`}
\begin{ocamldocdescription}
regresa el peso de la arsita entre dos nodos a partir del indice de ambos


\end{ocamldocdescription}




\label{val:Grafica.get-underscorevecinos}\begin{ocamldoccode}
val get_vecinos : ciudad -> int array
\end{ocamldoccode}
\index{get-underscorevecinos@\verb`get_vecinos`}
\begin{ocamldocdescription}
funcion para obtener los vecinos de un verice

{\bf Returns }arreglo de indice de vecinos


\end{ocamldocdescription}


\section{Module {\tt{Solucion}} : Modulo para operar las soluciones}
\label{module:Solucion}\index{Solucion@\verb`Solucion`}




\ocamldocvspace{0.5cm}



\label{val:Solucion.inicializa}\begin{ocamldoccode}
val inicializa : int -> unit
\end{ocamldoccode}
\index{inicializa@\verb`inicializa`}
\begin{ocamldocdescription}
funcion para inicializar el generador de numeros seudo aleatorios con la semilla


\end{ocamldocdescription}




\label{val:Solucion.ant1}\begin{ocamldoccode}
val ant1 : int Pervasives.ref
\end{ocamldoccode}
\index{ant1@\verb`ant1`}
\begin{ocamldocdescription}
variable que guarda la primera posicion del cambio anterior que se genero
en la funcion vecino


\end{ocamldocdescription}




\label{val:Solucion.ant2}\begin{ocamldoccode}
val ant2 : int Pervasives.ref
\end{ocamldoccode}
\index{ant2@\verb`ant2`}
\begin{ocamldocdescription}
variable que guarda la segunda posicion del cambio anterior que se genero
en la funcion vecino


\end{ocamldocdescription}




\label{val:Solucion.max-underscoredist-underscorepath}\begin{ocamldoccode}
val max_dist_path : float Pervasives.ref
\end{ocamldoccode}
\index{max-underscoredist-underscorepath@\verb`max_dist_path`}
\begin{ocamldocdescription}
variable que guarda la maxima distancia entre dos ciudades
Se debe inicializar con la funcion init\_max\_avg


\end{ocamldocdescription}




\label{val:Solucion.avg-underscorepath}\begin{ocamldoccode}
val avg_path : float Pervasives.ref
\end{ocamldoccode}
\index{avg-underscorepath@\verb`avg_path`}
\begin{ocamldocdescription}
variable que guarda la distancia promedio entre dos ciudades
Se debe inicializar con la funcion init\_max\_avg


\end{ocamldocdescription}




\label{val:Solucion.vecino}\begin{ocamldoccode}
val vecino : {\textquotesingle}a array -> unit
\end{ocamldoccode}
\index{vecino@\verb`vecino`}
\begin{ocamldocdescription}
funcion que calcula el "vecino" de una solucion
La funcion solo realiza un intercambio de valores y guarda los indices de los
valores intervambiados para poder volver a la solucion de que se partio


\end{ocamldocdescription}




\label{val:Solucion.regresa}\begin{ocamldoccode}
val regresa : {\textquotesingle}a array -> unit
\end{ocamldoccode}
\index{regresa@\verb`regresa`}
\begin{ocamldocdescription}
funcion que regresa el cambio hecho por vecino
Solo funciona para una anterior


\end{ocamldocdescription}




\label{val:Solucion.permuta}\begin{ocamldoccode}
val permuta : {\textquotesingle}a array -> unit
\end{ocamldoccode}
\index{permuta@\verb`permuta`}
\begin{ocamldocdescription}
funcion para conseguir una permutacion aleatoria de la instancia inicializa
y usarla como solucion inicial de la heuristica


\end{ocamldocdescription}




\label{val:Solucion.init-underscoremax-underscoreavg}\begin{ocamldoccode}
val init_max_avg : Grafica.ciudad array -> int array -> unit
\end{ocamldoccode}
\index{init-underscoremax-underscoreavg@\verb`init_max_avg`}
\begin{ocamldocdescription}
Funcion que inicializa la distancia promedio y distancia maxima entre dos ciudades
en la solucion


\end{ocamldocdescription}




\label{val:Solucion.f}\begin{ocamldoccode}
val f : Grafica.ciudad array -> int array -> float
\end{ocamldoccode}
\index{f@\verb`f`}
\begin{ocamldocdescription}
funcion de costo de la heuristica


\end{ocamldocdescription}




\label{val:Solucion.genera-underscoresolucion}\begin{ocamldoccode}
val genera_solucion : Grafica.ciudad array -> int -> int array
\end{ocamldoccode}
\index{genera-underscoresolucion@\verb`genera_solucion`}
\begin{ocamldocdescription}
funcion para obtendran una instancia aleatoria que tenga una solucion almenos


\end{ocamldocdescription}




\label{val:Solucion.factible}\begin{ocamldoccode}
val factible : Grafica.ciudad array -> int array -> int * bool
\end{ocamldoccode}
\index{factible@\verb`factible`}
\begin{ocamldocdescription}
funcion que indica si una solucion es factible y el numero de desconexiones 
presenta


\end{ocamldocdescription}


\section{Module {\tt{Svg}} : Modulo para generar las graficas del la solucion y el comportamiento de las evaluaciones}
\label{module:Svg}\index{Svg@\verb`Svg`}




\ocamldocvspace{0.5cm}



\label{val:Svg.file}\begin{ocamldoccode}
val file : string -> string -> string
\end{ocamldoccode}
\index{file@\verb`file`}




\label{val:Svg.creaLinea}\begin{ocamldoccode}
val creaLinea :
  Pervasives.out_channel -> float -> float -> float -> float -> string -> unit
\end{ocamldoccode}
\index{creaLinea@\verb`creaLinea`}




\label{val:Svg.creaLineaf}\begin{ocamldoccode}
val creaLineaf :
  Pervasives.out_channel -> float -> float -> float -> float -> string -> unit
\end{ocamldoccode}
\index{creaLineaf@\verb`creaLineaf`}




\label{val:Svg.creaCirculo}\begin{ocamldoccode}
val creaCirculo :
  Pervasives.out_channel -> float -> float -> int -> string -> unit
\end{ocamldoccode}
\index{creaCirculo@\verb`creaCirculo`}




\label{val:Svg.creaCirculof}\begin{ocamldoccode}
val creaCirculof :
  Pervasives.out_channel -> float -> float -> int -> string -> unit
\end{ocamldoccode}
\index{creaCirculof@\verb`creaCirculof`}




\label{val:Svg.encabezado}\begin{ocamldoccode}
val encabezado : Pervasives.out_channel -> int -> int -> unit
\end{ocamldoccode}
\index{encabezado@\verb`encabezado`}




\label{val:Svg.svg}\begin{ocamldoccode}
val svg : Pervasives.out_channel -> Grafica.ciudad array -> int array -> unit
\end{ocamldoccode}
\index{svg@\verb`svg`}




\label{val:Svg.guarda}\begin{ocamldoccode}
val guarda : Grafica.ciudad array -> int array -> int -> unit
\end{ocamldoccode}
\index{guarda@\verb`guarda`}




\label{val:Svg.gfuncion}\begin{ocamldoccode}
val gfuncion : (int, float) Hashtbl.t -> int -> unit
\end{ocamldoccode}
\index{gfuncion@\verb`gfuncion`}


\section{Module {\tt{Umbrales}} : Modulo principal que implementa la heristica de aceptacion por umbrales}
\label{module:Umbrales}\index{Umbrales@\verb`Umbrales`}




\ocamldocvspace{0.5cm}



\label{val:Umbrales.mejor-underscoresolucion}\begin{ocamldoccode}
val mejor_solucion : int array Pervasives.ref
\end{ocamldoccode}
\index{mejor-underscoresolucion@\verb`mejor_solucion`}
\begin{ocamldocdescription}
Variable que guarda la solucion mejor evaluada de durante todo el proceso


\end{ocamldocdescription}




\label{val:Umbrales.mejor-underscorefs}\begin{ocamldoccode}
val mejor_fs : float Pervasives.ref
\end{ocamldoccode}
\index{mejor-underscorefs@\verb`mejor_fs`}
\begin{ocamldocdescription}
Variable que guarda la mejor evaluacion de una solucion


\end{ocamldocdescription}




\label{val:Umbrales.evals}\begin{ocamldoccode}
val evals : (int, float) Hashtbl.t
\end{ocamldoccode}
\index{evals@\verb`evals`}
\begin{ocamldocdescription}
diccionario que guarda las evaluaciones de soluciones aceptadas


\end{ocamldocdescription}




\label{val:Umbrales.aceptadas}\begin{ocamldoccode}
val aceptadas : int Pervasives.ref
\end{ocamldoccode}
\index{aceptadas@\verb`aceptadas`}
\begin{ocamldocdescription}
variable que guarda el numero de soluciones aceptadas


\end{ocamldocdescription}




\label{val:Umbrales.calcula-underscorelote}\begin{ocamldoccode}
val calcula_lote :
  Grafica.ciudad array -> float -> int array -> float * int array
\end{ocamldoccode}
\index{calcula-underscorelote@\verb`calcula_lote`}
\begin{ocamldocdescription}
Funcion que va construyendo los lotes y regresando el promedio de la evaluacion de la funcion de costo


\end{ocamldocdescription}




\label{val:Umbrales.aceptacion-underscorepor-underscoreumbrales}\begin{ocamldoccode}
val aceptacion_por_umbrales :
  Grafica.ciudad array -> float -> int array -> unit
\end{ocamldoccode}
\index{aceptacion-underscorepor-underscoreumbrales@\verb`aceptacion_por_umbrales`}
\begin{ocamldocdescription}
funcion principal para la heristica


\end{ocamldocdescription}




\label{val:Umbrales.conecta-underscoreciudad}\begin{ocamldoccode}
val conecta_ciudad : Grafica.ciudad array -> string array -> {\textquotesingle}a -> unit
\end{ocamldoccode}
\index{conecta-underscoreciudad@\verb`conecta_ciudad`}
\begin{ocamldocdescription}
funcion que se aplicara a cada valor que se regrese de la tabla "connections" y a partir de cada fila conecta dos ciudades


\end{ocamldocdescription}




\label{val:Umbrales.agrega-underscoreciudad}\begin{ocamldoccode}
val agrega_ciudad : Grafica.ciudad array -> string array -> {\textquotesingle}a -> unit
\end{ocamldoccode}
\index{agrega-underscoreciudad@\verb`agrega_ciudad`}
\begin{ocamldocdescription}
funcion que se aplicara a cada valor que se regrese de la tabla "cities" y a partir de cada fila de esta se crea una ciudad de la grafica


\end{ocamldocdescription}




\label{val:Umbrales.umbrales}\begin{ocamldoccode}
val umbrales : Grafica.ciudad array -> int array -> int -> unit
\end{ocamldoccode}
\index{umbrales@\verb`umbrales`}
\begin{ocamldocdescription}
funcion que realiza el preproceso para la heristica y registra los resultados


\end{ocamldocdescription}




\label{val:Umbrales.construye-underscoregrafica}\begin{ocamldoccode}
val construye_grafica : unit -> Grafica.ciudad array
\end{ocamldoccode}
\index{construye-underscoregrafica@\verb`construye_grafica`}
\begin{ocamldocdescription}
Funcion que a partir de la base de datos predefinida construye la grafica de ciudades

{\bf Returns }la grafica de ciudades para el problema tsp


\end{ocamldocdescription}




\label{val:Umbrales.lee-underscoresolucion}\begin{ocamldoccode}
val lee_solucion : unit -> int array
\end{ocamldoccode}
\index{lee-underscoresolucion@\verb`lee_solucion`}
\begin{ocamldocdescription}
Funcion que lee el archivo que se pasa como argumento y construye solucion inicial
	con la que se trabajara la heuristica


\end{ocamldocdescription}


\end{document}